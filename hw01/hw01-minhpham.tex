\documentclass[13pt]{letter}
\usepackage[utf8]{inputenc}
\usepackage{amsmath}
\usepackage{amsfonts}
\usepackage{amssymb}
\usepackage[T1]{fontenc}
\linespread{1.2}
\begin{document}
Student: Minh Pham \\
EID: mlp2279

\begin{center}
	\textbf{Homework 1}
\end{center}

\begin{enumerate}
	\item (10 points)
	Given:
	\begin{verbatim}
	glutInitWindowSize(800, 600);
	gluOrtho2D(-100.0, 100.0, -100.0, 100.0);	
	\end{verbatim}

	Convert the following object coordinates to window coordinates. \\
	
	Solution:
	\begin{tabbing}
	\hspace{20pt} \= Object Coordinates \= Window Coordinates \\
	(a)           \> (0, 0)             \> (400, 300) \\
	(b)           \> (-50, 50)          \> (200, 150) \\
	(c)	          \> (-75, -100)        \> (100, 600) \\
	(d)	          \> (90, 10)           \> (760, 270) \\
	(e)	          \> (0, -40)           \> (400, 420) \\
	\end{tabbing}
	

	\item (5 points)
	Let:
	\begin{enumerate}
	\item[•] $\alpha$, $\beta$, $\gamma$ be scalars
	\item[•] $A$, $B$, $C$ be points
	\item[•] $u$, $v$, $w$ be vectors
	\end{enumerate}
	
	Are the following operations defined? \\
	Answer T/F/? if operation is defined/undefined/don't know. \\

	\begin{tabbing}
	\hspace{20pt} \= Operation \hspace{20pt} \= Defined?  \\
	(a)           \> $v - u$                 \> T \\
	(b)           \> $v - A$                 \> F \\
	(c)           \> $A - v$                 \> T \\
	(d)           \> $A + \alpha(B-A)$       \> T \\
	(e)           \> $\alpha A + v$          \> F \\
	\end{tabbing}
	
	\item (5 points)
	Find a homogeneous-coordinate representation of a plane. \\
	
	The following solution assumes we are working in 3 dimensions. \\
	Suppose we had a point $P(x_0, y_0, z_0, 1)$ and two non-parallel vectors $a$ and $b$. These completely define a plane in 3 dimensions. Let $n(x_n, y_n, z_n, 0)$ be the cross product of $a$ and $b$. Then $n$ is a vector perpendicular to the plane defined by $P$, $a$ and $b$. \\
	Let $Q(x, y, z, 1)$ be an arbitrary point in 3 dimensions. \\
	Let $v(x-x_0, y-y_0, z-z_0, 0)$ be the vector from $P$ to $Q$. $Q$ lies on the plane defined by $P$, $a$, and $b$ if and only if $v$ is a linear combination of $a$ and $b$. This implies that $v$ is perpendicular to $n$, since $n$ is the cross-product of the two. \\
	Because $v$ and $n$ are perpendicular, their dot product must be 0. \\
	A homogeneous-coordinate representation of a plane is therefore:
	\begin{enumerate}
		\item[] $n \cdot v = 0$
    \item[$\Leftrightarrow$] $x_n * (x-x_0) + y_n * (y-y_0) + z_n * (z-z_0) = 0$
	\end{enumerate}
	
	\item (15 points)
	If we are interested in only two-dimensional graphics, we can use three-dimensional homogeneous coordinates by representing a point as $p = [x~y~1]^T$ and a vector as $v = [a~b~0]^T$. \\
	Find the 3x3 rotation, translation, scaling, and shear matrices $R$, $T$, $S$, and $H$, respectively. How many degrees of freedom are there in an affine transformation for transforming two-dimensional points?
	
	In two dimensions we have:
	\begin{enumerate}
		\item[•]	 Rotation matrix for rotating an angle $\theta$ counter-clockwise: \\
		$R = \left[
		\begin{array}{ccc}
			\cos	\theta & -\sin\theta  & 0 \\
			\sin\theta & \cos\theta   & 0 \\
			0          & 0            & 1
		\end{array}
		\right]$
		\item[•] Translation matrix for translating by a vector $[\alpha_x~\alpha_y~0]^T$ : \\
		$T = \left[
		\begin{array}{ccc}
			1 & 0 	& \alpha_x \\
			0 & 1 & \alpha_y \\
			0 & 0 & 1
		\end{array}
		\right]$
		\item[•] Scaling matrix for scaling by $\beta_x$ in the $x$ direction, and $\beta_y$ in the $y$ direction: \\
		$S = \left[
		\begin{array}{ccc}
			\beta_x & 0       & 0 \\
			0       & \beta_y & 0 \\
			0       & 0       & 1
		\end{array}
		\right]$
		\item[•] Shear matrix for shearing along the $x$ axis by the angle $\theta$: \\
		$H_x = \left[
		\begin{array}{ccc}
			1 & \cot\theta & 0 \\
			0 & 1          & 0 \\
			0 & 0          & 1
		\end{array}
		\right]$	
		\item[$\cdot$] Shear matrix for shearing along the $y$ axis by the angle $\theta$: \\
		$H_y = \left[
		\begin{array}{ccc}
			1          & 0 & 0 \\
			\cot\theta & 1 & 0 \\
			0          & 0 & 1
		\end{array}
		\right]$	
	\end{enumerate}
	
	In two dimentions there are 6 degrees of freedom for affine transformations, represented by the top and middle rows in the transformation matrix (thhe bottom row is always 0 0 1 for 2 dimension).
	
	\item (15 points)
	Derive a rotation matrix where we rotate first about the x-axis $R_x(\theta_x)$, then about the y-axis $R_y(\theta_y)$, and then about the z-axis $R_z(\theta_z)$. Assume that the fixed point is the origin.
	
	Assuming we are working in 3 dimension, in a right-handed coordinate system. \\
	We have the following rotation matrices around each of the axes:

	$$R_x = \left[
	\begin {array}{cccc}
	1 & 0           & 0           & 0 \\
	0 & \cos\theta  & -\sin\theta & 0 \\
	0 & \sin\theta  & \cos\theta  & 0 \\
	0 & 0           & 0           & 1
	\end {array}
	\right]$$
	
	$$R_y = \left[
	\begin {array}{cccc}
	\cos\theta  & 0 & \sin\theta  & 0 \\
	0           & 1 & 0           & 0 \\
	-\sin\theta & 0 & \cos\theta  & 0 \\
	0           & 0 & 0           & 1
	\end {array}
	\right]$$

	$$R_z = \left[
	\begin {array}{cccc}
	\cos\theta & -\sin\theta & 0 & 0 \\
	\sin\theta & \cos\theta  & 0 & 0 \\
	0          & 0           & 1 & 0 \\
	0          & 0           & 0 & 1
	\end {array}
	\right]$$

	The transformation that rotates a vector $v$ first about the x-axis by $\theta_x$, then about the y-axis by $\theta_y$, and then about the z-axis by $\theta_z$ is $R_z(\theta_z)R_y(\theta_y)R_x(\theta_x)v$. \\
	We get the combined transformation matrix $M$ by multiplying the three individual rotation matrices together. We have:
	\begin{enumerate}
		\item[] $M = R_z(\theta_z)R_y(\theta_y)R_x(\theta_x)$
		\item[] $M = \left[ 	\begin {array}{cccc}
													\cos\theta_z & -\sin\theta_z & 0 & 0 \\
													\sin\theta_z & \cos\theta_z  & 0 & 0 \\
													0          & 0           & 1 & 0 \\
													0          & 0           & 0 & 1
						     \end {array} 	\right]
							   \left[ 	\begin {array}{cccc}
													\cos\theta_y  & 0 & \sin\theta_y  & 0 \\
													0             & 1 & 0             & 0 \\
													-\sin\theta_y & 0 & \cos\theta_y  & 0 \\
													0             & 0 & 0             & 1
                 \end {array} 	\right]
	               \left[ 	\begin {array}{cccc}
													1 & 0           & 0           & 0 \\
													0 & \cos\theta_x  & -\sin\theta_x & 0 \\
													0 & \sin\theta_x  & \cos\theta_x  & 0 \\
													0 & 0           & 0           & 1
	               \end {array} \right]$
		\item[] $M = \left[ 	\begin {array}{cccc}
													\cos\theta_z\cos\theta_y & -\sin\theta_z & \cos\theta_z\sin\theta_y & 0 \\
													\sin\theta_z\cos\theta_y & \cos\theta_z  & \sin\theta_z\sin\theta_y & 0 \\
													-\sin\theta_y            & 0           & \cos\theta_y & 0 \\
													0                        & 0           & 0 & 1
						     \end {array} 	\right]
						     \left[ 	\begin {array}{cccc}
													1 & 0           & 0           & 0 \\
													0 & \cos\theta_x  & -\sin\theta_x & 0 \\
													0 & \sin\theta_x  & \cos\theta_x  & 0 \\
													0 & 0           & 0           & 1
	               \end {array} \right]$
		\item[] $M = \left[ 	\begin {array}{cccc}
													\cos\theta_z\cos\theta_y & -\sin\theta_z\cos\theta_x + \cos\theta_z\sin\theta_y\sin\theta_x & \sin\theta_z\sin\theta_x + \cos\theta_z\sin\theta_y\cos\theta_x & 0 \\
													\sin\theta_z\cos\theta_y & \cos\theta_z\cos\theta_x + \sin\theta_z\sin\theta_y\sin\theta_x  &  -                                      \cos\theta_z\sin\theta_x + \sin\theta_z\sin\theta_y\cos\theta_x & 0 \\
													-\sin\theta_y & \cos\theta_y\sin\theta_x & \cos\theta_y\cos\theta_x & 0 \\
													0 & 0           & 0           & 1
	               \end {array} \right]$
	\end{enumerate}

\end{enumerate}
\end{document}